\documentclass[12pt, a4paper]{book}
\usepackage[T2A]{fontenc}
\usepackage[utf8]{inputenc}
\usepackage[russian]{babel}
\usepackage{fancyhdr}
\usepackage{multicol}
\usepackage[topmargin=-20pt,headsep=10pt, textheight=27cm, width=20cm]{geometry}
\fancypagestyle{firststyle} 
{
\fancyhf{}
\fancyhead[L]{\textbf{\small{По страницам школьных учебников}}}
\fancyfoot[L]{36}
}
\setlength\columnsep{30pt}
\begin{document}
\thispagestyle{firststyle}


\begin{multicols}{2}

\
\newline
\newline
\newline
\newline
\newline
\newline
\newline
\newline
\newline
\newline
Г. Перевалов
\
\newline
\newline
\newline
\section*{Можно и без производной}



Преобразование графика
На вступительных экзаменах одному
из абитуриентов было предложено 
построить график функции 
Он, как рекомендует учебное пособие "Алгебра и начала
анализа 9-10" (п. 27), нашел 
область определения функции f, вычислил производную f, увидел, что
она всюду на D(f) положительна,
сделал вывод, что функция f на D(f)
возрастает, нашел точку пересечения 
графика с осью абсуисс, записал 
результаты исследования в виде 
\columnbreak
таблицы и построил график
(рис. 1, a).
Однако искомый график можно
построить без всяких вычислений,
если применить правила пребразо-
вания графиков, изложенные в конце
упомятутого пособия ("Материал
для повторения", п. 9). Перечислим 
эти правила:
1. График функции  
получается из графика функции
переносом , то есть
переносом параллельно оси ординат
на B - вверх, если b>0; вниз, 
если b<0 (рис. 2).
2. График функции у=f(x+b)
получается из графика функции
у = 1 (х) переносом 5 (-b; 0), то есть
переносом параллельно оси абсцисс
на —в - влево, если b>0; вправо,
если b<0 (рис. 3.).
3. График функции у= А • f(x)
получается умножением каждой ор-
динаты графика функции у=/(x)
на А, то есть растяжением от оси
абсцисс в А раз, если А > 1, и сжатием
к оси абсцисс в 1/A раз, если 0<А<1
(рис. 4)
3'. График функции у=-/(x)
получается симметрией графика
функции у=f(x) относительно оси
абсцисс (рис. 5).
4. График функции у=/(ax)
получается сжатием графика функ-
ции у=/(х) к оси ординат в а раз,
если а>1, и растяжением от оси
ординат в раз, если 0<а<1
(рис. 6).
4'График функции у=[(-х)
получается симметрией графика
функции у=/(х) относительно оси
ординат (рис. 7).
Заодно уж приведем два сораздо
менее важных, менее универсальных
правила, которые могут пригодиться
в «абитуриентских» задачах:
5.График Функции
совпадает с графиком функции 
там, где , и полу-
чается из него симметрией относи-
тельно оси абсцисс там, где /(x) <0
(рис. 8).
6. График функции у =1(|*|) при
×> 0 совпадает с графиком функции
y=f(x); при х<0 он получается
\end{multicols}
\end{document}
